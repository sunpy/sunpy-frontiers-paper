\section{Conclusion}
\label{sec:conclusion}

In this paper, we have summarized the \sunpyproj and its various components, including the code developed and maintained by the project (\autoref{sec:code}), the people that comprise the project (\autoref{sec:people}) and the community that the project serves (\autoref{sec:community}).
In particular, we have discussed how the \sunpypkg package and the wider set of affiliated packages form an ecosystem for performing solar physics research in Python and the types of analysis that such an ecosystem enables (see \autoref{fig:affiliated-package-showcase}).
Finally, we have summarized a tentative roadmap to steer the direction of the \sunpyproj in the coming years.
Importantly, we hope that such a high level description will provide a more clear understanding of the \sunpyproj and the wider ecosystem and will encourage contributions of all forms from the global solar physics community.

\added[id=NF]{We want to end on an explicit call to instrument teams to consider developing affiliated packages.
The \sunpyproj provides a number of benefits to instrument teams, starting off with a community of developers and users who will provide feedback on the development of your package.
While we are still primarily a team of volunteers, we can provide (limited) code review, Python or packaging support.
In addition, a Python package template is being written to simplify the Python packaging process and will enable automatic updates as changes are made.
With this in place, we envision a future where creating and maintaining scientific solar physics packages is not a hassle and would allow a larger ecosystem.}
