\section{Community}
\label{section:community}

A major part of the sustainability, growth, and usage of \sunpypkg and the affiliated packages is fostering an active community within the solar physics and heliophysics research community and cultivating collaborations with developers maintaining packages in the wider scientific Python community. 

\subsection{Engagement with the Solar Physics Community}
\label{sec:communication}

In order for the \sunpyproj to maintain and grow the \sunpypkg core package and affiliated packages within the ecosystem, engagement with the wider solar physics community is critical. 
The mission of the \sunpyproj is to be community-led, and the development is driven by the needs of the solar physics community. 
To facilitate this, the \sunpyproj is building a community for which there is inclusive and open communication between those developing \sunpypkg and those using \sunpypkg in their scientific research. 
Active contributions from users in terms of bug reports, issues encountered with code or documentation, and feature requests are all vital to the sustainability and future of the \sunpyproj.
We emphasis that being part of the \sunpyproj does not necessarily mean writing software. Contributions in the form of feedback and suggestions are equally as important. 

To foster communication, the \sunpyproj supports several communication platforms (see \autoref{sec:communication_channels}) through which users and developers can regularly interact. 
The \sunpyproj posts on solar physics noticeboards about recent releases and big changes, and regularly advertises \sunpypkg and affiliated packages at scientific conferences, providing tutorials and support. 
We also ask that if \sunpypkg is used for scientific work that it is cited in the literature\footnote{See this page for a guide on how to cite \sunpypkg in published works: \url{https://sunpy.org/about}}, thereby increasing its visibility to the scientific community and ultimately contributing to the continued growth and development of the package.
More recently, the \sunpyproj has improved communications and established relationships with data providers such as VSO and the SOAR, and teams supporting both operating and developing instruments and missions. 
The \sunpyproj is always looking for ways to improve the accessibility of the project and to grow the community, and as such, suggestions are always welcome. 

\subsubsection{Communication Channels}
\label{sec:communication_channels}
Over the years, the usage of \sunpypkg and affiliated packages within the solar physics community has increased, and with that methods to communicate within the SunPy community have also increased. 
At the time of writing, several distinct communication channels are available.
These include:

\begin{itemize}
    \item Multiple GitHub repositories for bug reports and feature requests. These are listed under the sunpy GitHub organization: \url{https://github.com/sunpy}/.
    \item Real time messaging on Matrix at \url{https://matrix.to/#/#sunpy:openastronomy.org}.
    \item Mailing Lists at \url{https://groups.google.com/g/sunpy}
    \item An online community forum at \url{https://community.openastronomy.org/}
    \item Weekly public calls that anyone can participate in.
\end{itemize}

Each has their own distinct purpose, and was created as a need arose for their existence.
For example, the GitHub repository is used for the development of \sunpypkg and issues and bugs can be raised there.
However some scientists may not be familiar with GitHub and would like to ask a general question on how to do something within \sunpypkg. 
For this, the mailing list, community forum, or real time Matrix chat may be the most appropriate.
We actively encourage users and those interested in contributing to use any or all of these communication channels. 

In addition to the main communication platforms specific to the \sunpyproj, we maintain a presence within other communication channels used by the wider heliophysics community, including Helionauts\footnote{\url{https://helionauts.org/}} and communication channels used by the Python in Heliophysics Community.

\subsection{Python in Heliophysics Community (PyHC)}

The Python in Heliophysics Community (PyHC)\footnote{\url{https://heliopython.org/}} \citep{barnum2022python} is a project with similar goals as the \sunpyproj, but focuses on the wider Heliophysics community \citep{https://doi.org/10.1029/2018JA025877}.
These include providing coding standards \citep{annex_a_2018_2529131}, curating a list of participating projects\footnote{\url{https://heliopython.org/projects/}}, hosting bi-monthly community meetings, and organizing an inaugural summer school for early career researchers.
The \sunpyproj is actively involved in PyHC, with \sunpypkg being one of the core PyHC packages.
\sunpyproj members regularly attend community meetings and present updates.
The \sunpyproj also took part in the PyHC 2022 summer school.
Moving forward, PyHC and the \sunpyproj will continue to collaborate and build upon efforts of using \sunpypkg and affiliated packages within the larger heliophysics Python ecosystem.

\subsection{Collaboration with the Wider Python Ecosystem}

The \sunpypkg package forms part of the wider Python scientific ecosystem, requiring active collaboration with other scientific Python packages.
Whenever possible, we aim to contribute to relevant open-source projects rather than duplicating functionality.
As an example, large parts of \sunpypkg depend on core functionality developed in the \astropypkg package, including support for handling units, times, and coordinates. 

The \sunpyproj is sponsored by NumFOCUS, ``a nonprofit supporting open code for better science''\footnote{https://numfocus.org/}.
NumFOCUS provides financial and organizational support for several important packages (e.g., \package{numpy}, \package{pandas} and \package{xarray}) and facilities collaborate between packages throughout the scientific Python ecosystem. 
One example of this is the annual NumFOCUS summit that brings together the leaders of these packages to discuss interoperability, funding sources and other high-level topics that improve the Python ecosystem as a whole.

In addition, the \sunpyproj is a member of the OpenAstronomy organization\footnote{\url{https://openastronomy.org/}}.
OpenAstronomy was created to collaborate on outreach, organize conferences such as Python in Astronomy, develop common tooling for infrastructure, and apply to internship programs such as Google Summer of Code (GSoC)\footnote{\url{https://summerofcode.withgoogle.com/}} and Outreachy\footnote{\url{https://www.outreachy.org/}}.
GSoC has been an invaluable source of programming effort for the \sunpyproj over the past decade.
The contributions from participants in this program been crucial to \sunpypkg.
Examples of successful projects include the conversion to using \package{astropy.time} module and creating a new Python API wrapper for Helioviewer.org\footnote{\url{https://hvpy.readthedocs.io/}}.
As the focus of the OpenAstronomy organization is the broader astrophysics and astronomy community, \sunpyproj's participation has enabled closer ties within the rapidly growing astronomy Python landscape.
