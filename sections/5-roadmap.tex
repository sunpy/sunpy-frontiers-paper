\section{The Future of The \sunpyproj}
\label{sec:the-future-of-the-sunpyproj}

The \sunpyproj has incorporated sustainability as a core principle: code is publicly available and openly-developed by the solar physics community.
This development is driven by, and for, the solar physics community, responding to the needs of researchers for data analysis tools and techniques, and software for working with data from new missions.
This means both the \sunpypkg core package and other affiliated packages are continually changing and expanding.
In September 2022 several members of the \sunpyproj met at a coordination meeting to discuss the future of the project.
Two key areas that emerged were updating the governance structure, and creating a roadmap for future development.
The roadmap provides:
\begin{enumerate}
    \item a set of priorities for developers to work on in the medium term.
    \item a well scoped list of work items that funding can be sought for.
    \item a mechanism to solicit input from the wider solar physics community on the medium term priorities from \sunpyproj.
\end{enumerate}

At the time of writing, items in the draft roadmap include:
\begin{itemize}
    \item Improving support and functionality for data with spectral coordinates (e.g., rastering spectrometers)
    \item Enabling multi-dimensional data sets (i.e., beyond 2D images).
    \item Improving support for running \sunpypkg on cloud infrastructure.
    \item Creating a consistently structured set of documentation across all the \sunpyproj packages.
    \item Adding functionality to rapidly visualize large data sets.
    \item Restructuring the project governance to, among other things, transform the lead developer positions into a multi-person steering committee and create an ombudsperson role.
\end{itemize}

The next step is consultation with the wider solar physics community.
We invite feedback on this proposed roadmap via any of the aforementioned communication channels (see \autoref{sssec:communication-channels}) or by opening an issue on the repository used for tracking high-level, project-wide tasks and suggestions\footnote{\url{https://github.com/sunpy/sunpy-project}}.
